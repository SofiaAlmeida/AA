%%%
% Plantilla de Memoria
% Modificación de una plantilla de Latex de Nicolas Diaz para adaptarla 
% al castellano y a las necesidades de escribir informática y matemáticas.
%
% Editada por: Mario Román
%
% License:
% CC BY-NC-SA 3.0 (http://creativecommons.org/licenses/by-nc-sa/3.0/)
%%%

%%%%%%%%%%%%%%%%%%%%%%%%%%%%%%%%%%%%%%%%%
% Thin Sectioned Essay
% LaTeX Template
% Version 1.0 (3/8/13)
%
% This template has been downloaded from:
% http://www.LaTeXTemplates.com
%
% Original Author:
% Nicolas Diaz (nsdiaz@uc.cl) with extensive modifications by:
% Vel (vel@latextemplates.com)
%
% License:
% CC BY-NC-SA 3.0 (http://creativecommons.org/licenses/by-nc-sa/3.0/)
%
%%%%%%%%%%%%%%%%%%%%%%%%%%%%%%%%%%%%%%%%%

%----------------------------------------------------------------------------------------
%	PAQUETES Y CONFIGURACIÓN DEL DOCUMENTO
%----------------------------------------------------------------------------------------

%%% Configuración del papel.
% microtype: Tipografía.
% mathpazo: Usa la fuente Palatino.
\documentclass[a4paper, 20pt]{article}
\usepackage[a4paper,margin=1in]{geometry}
\usepackage[protrusion=true,expansion=true]{microtype}
\usepackage{mathpazo}

% Indentación de párrafos para Palatino
\setlength{\parindent}{0pt}
  \parskip=8pt
\linespread{1.05} % Change line spacing here, Palatino benefits from a slight increase by default


%%% Castellano.
% noquoting: Permite uso de comillas no españolas.
% lcroman: Permite la enumeración con numerales romanos en minúscula.
% fontenc: Usa la fuente completa para que pueda copiarse correctamente del pdf.
\usepackage[spanish,es-noquoting,es-lcroman,es-tabla,,es-nodecimaldot]{babel}
\usepackage[utf8]{inputenc}
\usepackage{fontenc}
\selectlanguage{spanish}


%%% Gráficos
\usepackage{graphicx} % Required for including pictures
\usepackage{wrapfig} % Allows in-line images
\usepackage[usenames,dvipsnames]{color} % Coloring code
%\usepackage{subcaption}
\usepackage{subfig}
\graphicspath{{./fig/}}


%%% Matemáticas
\usepackage{amsmath}
\usepackage{physics} % para las derivadas parciales
\usepackage[Symbol]{upgreek} %pi

%%% Pseudocódigo
\usepackage{algorithmicx}
\usepackage[ruled]{algorithm}
\usepackage{algpseudocode}

\newcommand{\alg}{\texttt{algorithmicx}}
\newcommand{\old}{\texttt{algorithmic}}
\newcommand{\euk}{Euclid}
\newcommand\ASTART{\bigskip\noindent\begin{minipage}[b]{0.5\linewidth}}
\newcommand\ACONTINUE{\end{minipage}\begin{minipage}[b]{0.5\linewidth}}
\newcommand\AENDSKIP{\end{minipage}\bigskip}
\newcommand\AEND{\end{minipage}}

%%% Código
\usepackage{listings}

%%% Tablas
\usepackage{tabularx}
\usepackage{float}
\usepackage{adjustbox}
\usepackage{booktabs}

% Enlaces y colores
\usepackage{hyperref}
\usepackage[dvipsnames]{xcolor}
\definecolor{webgreen}{rgb}{0,0.5,0}
\hypersetup{
  colorlinks=true,
  citecolor=RoyalBlue,
  urlcolor=RoyalBlue,
  linkcolor=RoyalBlue
}

%%% Bibliografía
\usepackage[backend=biber]{biblatex}
\DefineBibliographyStrings{spanish}{
  urlseen = {Último acceso}
}
\addbibresource{IN-P2.bib}


%%% Subsubsection con letras
\renewcommand{\thesubsubsection}{\thesubsection.\alph{subsubsection}}

%%% Itemize, enumitem
\usepackage{paralist}
\usepackage{enumitem}
%----------------------------------------------------------------------------------------
%	TÍTULO
%----------------------------------------------------------------------------------------
% Configuraciones para el título.
% El título no debe editarse aquí.
\renewcommand{\maketitle}{
  \begin{flushright} % Right align
  
  {\LARGE\@title} % Increase the font size of the title
  
  \vspace{50pt} % Some vertical space between the title and author name
  
  {\large\@author} % Author name
  \\\@date % Date
  \vspace{40pt} % Some vertical space between the author block and abstract
  \end{flushright}
}

%% Título
\title{\textbf{Título}\\ % Title
Subtítulo} % Subtitle

\author{\textsc{Autor1,\\Autor2} % Author
\\{\textit{Universidad de Granada}}} % Institution

\date{\today} % Date

%-----------------------------------------------------------------------------------------
%	DOCUMENTO
%-----------------------------------------------------------------------------------------

\begin{document}

%-----------------------------------------------------------------------------------------
%	TITLE PAGE
%-----------------------------------------------------------------------------------------

\begin{titlepage} % Suppresses displaying the page number on the title page and the subsequent page counts as page 1
	
	\raggedleft % Right align the title page
	
	\rule{1pt}{\textheight} % Vertical line
	\hspace{0.05\textwidth} % Whitespace between the vertical line and title page text
	\parbox[b]{0.8\textwidth}{ % Paragraph box for holding the title page text, adjust the width to move the title page left or right on the page
		
		{\Huge\bfseries Trabajo 2:\\[0.5\baselineskip] Programación\\[2\baselineskip]} % Title
		{\large\textit{Curso 2019/2020}\\[0.5\baselineskip]Aprendizaje Automático\\[1\baselineskip] }% Subtitle or further description
		{\Large\textsc{Sofía Almeida Bruno}\\[0.5\baselineskip]sofialmeida@correo.ugr.es} % Author name, lower case for consistent small caps
		
		\vspace{0.4\textheight} % Whitespace between the title block and the publisher
		
		{\noindent \\[0.5\baselineskip] }\\[\baselineskip] % Publisher and logo
	}

\end{titlepage}

%% Resumen (Descomentar para usarlo)
%\renewcommand{\abstractname}{Resumen} % Uncomment to change the name of the abstract to something else
%\begin{abstract}
% Resumen aquí
%\end{abstract}

%% Palabras clave
%\hspace*{3,6mm}\textit{Keywords:} lorem , ipsum , dolor , sit amet , lectus % Keywords
%\vspace{30pt} % Some vertical space between the abstract and first section


%% Índice
{\parskip=2pt
  \tableofcontents
}
\pagebreak

%%% Inicio del documento
%%%%%%%%%%%%%%%%%%%%%%%%%%%%%%%%%%%%%%%%%%%%%%%%%%%%%%%%%%%%%%%%%%%
%       EJERCICIO 1
%%%%%%%%%%%%%%%%%%%%%%%%%%%%%%%%%%%%%%%%%%%%%%%%%%%%%%%%%%%%%%%%%%%
\large
\section{Ejercicio sobre la búsqueda complejidad de H y el ruido}
El código correspondiente a este ejercicio se encuentra en el archivo \texttt{ej1.py}.
\subsection{Dibujar una gráfica con la nube de puntos de salida correspondiente.}
Para dibujar las gráficas con la nube de puntos se ha implementado la función \texttt{draw\_points} que toma como parámetro el vector $N$ de puntos.
\subsubsection{Considere $N = 50$, $dim = 2$, $rango = [-50, +50]$ con \texttt{simula\_unif(N, dim, rango)}.}
Llamando a la función \texttt{simula\_unif(N, dim, rango)} con los parámetros correspondintes, se generó el conjunto de puntos que podemos observar en la Figura \ref{fig:1a}. Notamos que, efectivamente, los puntos se han generado uniformemente en el rango.

\begin{figure}[H]
    \centering
    \includegraphics[width=0.75\textwidth]{points1a}
    \caption{Puntos generados con \texttt{simula\_unif}.}
    \label{fig:1a}
\end{figure}

\subsubsection{Considere $N = 50$ y $sigma = [5,7]$ con \texttt{simula\_gaus(N, dim, sigma)}.}
Ejecutando la función \texttt{simula\_gaus} con los parámetros pedidos se obtiene el conjunto de puntos mostrado en la Figura \ref{fig:1b}. En este caso los puntos se agrupan en torno al $(0,0)$ ya que la función usada para generarlos fue una gaussiana de media 0.

\begin{figure}[H]
    \centering
    \includegraphics[width=0.75\textwidth]{points1b}
    \caption{Puntos generados con \texttt{simula\_gaus}.}
    \label{fig:1b}
\end{figure}

\subsection{Con ayuda de la función \texttt{simula\_unif()} generar una muestra de puntos 2D a los que vamos añadir una etiqueta usando el signo de la función $f(x, y) = y - ax - b$, es decir el signo de la distancia de cada punto a la recta simulada con \texttt{simula\_recta()}.}
\subsubsection{Dibujar una gráfica donde los puntos muestren el resultado de su etiqueta, junto con la recta usada para ello. (Observe que todos los puntos están bien clasificados respecto de la recta).}

Se generan los puntos como en el caso anterior, en este caso tomamos una muestra de 500 puntos en el mismo rango. Se les asigna su clase mediante la función \texttt{f}, que toma los parámetros \texttt{a, b} obtenidos al ejecutar \texttt{simula\_recta}. Se ha utilizado la función \texttt{plot\_line} para generar la gráfica.

\begin{figure}[H]
    \centering
    \includegraphics[width=0.75\textwidth]{recta}
    \caption{Puntos generados con \texttt{simula\_unif}, clase asignada a partir del signo de la recta.}
    \label{fig:recta}
\end{figure}

En la Figura \ref{fig:recta} vemos las puntos generados coloreados según su clase y notamos cómo la recta es una frontera perfecta entre ambas. Esto es debido a que las etiquetas se asignaron en función al signo obtenido por la función que define la recta.

\subsubsection{Modifique de forma aleatoria un 10 \% etiquetas positivas y otro 10 \% de negativas y guarde los puntos con sus nuevas etiquetas. Dibuje de nuevo la gráfica anterior. (Ahora hay puntos mal clasificados respecto de la recta).}

En la función \texttt{modify\_classes} modificamos un 10\% de las etiquetas positivas y un 10\% de las etiquetas negativas. Pintamos la gráfica resultante, que podemos ver en la Figura \ref{fig:2b}. La gráfica es similar a la del apartado anterior, con la diferencia de que esta vez la recta no serpara correctamente a todos los puntos, al haber añadido ruido hay puntos que quedan mal clasificados.

\begin{figure}[H]
    \centering
    \includegraphics[width=0.75\textwidth]{recta+ruido}
    \caption{Puntos generados con \texttt{simula\_unif}, clase asignada a partir del signo de la recta y posterior adición de ruido al 10\% de las etiquetas de cada clase.}
    \label{fig:2b}
\end{figure}

\subsection{Supongamos ahora que las siguientes funciones definen la frontera de clasificación de los puntos de la muestra en lugar de una recta:}\linespread{-10}
\textbf{
  \begin{itemize}[topsep=-5pt]
  \item $f (x, y) = (x - 10)^2 + (y - 20)^2 - 400$
  \item $f (x, y) = 0,5(x + 10)^2 + (y - 20)^2 - 400$
  \item $f (x, y) = 0,5(x - 10)^2 - (y + 20)^2 - 400$
  \item $f (x, y) = y - 20x^2 - 5x + 3$
  \end{itemize}
Visualizar el etiquetado generado en 2b junto con cada una de las gráficas de cada una de las funciones. Comparar las formas de las regiones positivas y negativas de estas nuevas funciones con las obtenidas en el caso de la recta ¿Son estas funciones más complejas mejores clasificadores que la función lineal? ¿En que ganan a la función lineal? Explicar el razonamiento.}
\begin{figure}[H]
\def\tabularxcolumn#1{m{#1}}
\begin{tabularx}{\linewidth}{@{}cXX@{}}
%
\begin{tabular}{cc}
\subfloat[$f (x, y) = (x - 10)^2 + (y - 20)^2 - 400$.]{\includegraphics[width=0.5\textwidth]{f1}} 
   & \subfloat[$f (x, y) = 0,5(x + 10)^2 + (y - 20)^2 - 400$.]{\includegraphics[width=0.5\textwidth]{f2}}\\
\subfloat[$f (x, y) = 0,5(x - 10)^2 - (y + 20)^2 - 400$.]{\includegraphics[width=0.5\textwidth]{f3}} 
   & \subfloat[$f (x, y) = y - 20x^2 - 5x + 3$.]{\includegraphics[width=0.5\textwidth]{f4}}\\
\end{tabular}
\end{tabularx}

\caption{Fronteras de clasificación dadas por $f$.}\label{f:13}
\end{figure}

\newpage
%%%%%%%%%%%%%%%%%%%%%%%%%%%%%%%%%%%%%%%%%%%%%%%%%%%%%%%%%%%%%%%%%%%
%       REFERENCIAS
%%%%%%%%%%%%%%%%%%%%%%%%%%%%%%%%%%%%%%%%%%%%%%%%%%%%%%%%%%%%%%%%%%%
\printbibliography
\end{document}
